\documentclass[12pt, a4paper]{article}
\usepackage{indentfirst}
\usepackage{graphicx}
\usepackage{url}
\graphicspath{{images/}}
\title{GDPR: Is non-compliance profitable for Meta Platforms, Inc.?}
\author{Stoyan L. Kostadinov}
\date{June 2023}
\begin{document}
\maketitle

\begin{center}
    \includegraphics[width=0.3\textwidth]{fontys-logo}
\end{center}

\begin{abstract}
    % TODO
\end{abstract}

\subsection*{Keywords}

GDPR, Facebook, Meta, profit, non-compliance, big tech, privacy, financial
impact

\section*{Introduction}

In the recent years, the increasing usage of the Internet has lead to a rapid
expansion of data collection and processing\cite{khan2014big}. As concerns
regarding data privacy and protection grew, the European Union (EU) implemented
the General Data Protection Regulation (GDPR) in May
2018\cite{greengard2018weighing, historyGdpr}.

GDPR aimed to safeguard the privacy rights of individuals within the EU by
regulating the collection, storage, and transfer of personal
data\cite{historyGdpr}. Notably, data should be collected for specified,
explicit and legitimate purposes\cite{europeanParliamentGdprArticle5} and data
can be transferred outside the EU if general data processing principles are in
place \cite{europeanParliamentGdprArticle44}. Compliance with GDPR has become a
critical issue for big-tech companies, which often handle massive amounts of
user data.

One of those companies is \textit{Meta Platforms, Inc.} (formerly
\textit{Facebook, Inc.}), which has been fined numerous times for GDPR
violations\cite{mrevzar2023analysis}. This has raised questions about the
financial impact of GDPR non-compliance from \textit{Meta} and whether it is
profitable for the company to continue violating the regulation.

The goal of this paper is to investigate the financial impact of GDPR
non-compliance on \textit{Meta Platforms, Inc.} and to determine whether it is
profitable for the company to continue violating the regulations.

\section*{Method}

All numerical data about \textit{Meta Platforms, Inc.} has been gathered from
the official \textit{Meta Investor Relations} website\cite{fbMetaFinancials}.

GDPR fines imposed on \textit{Meta Platforms, Inc.} have been gathered from the
\textit{GDPR Enforcement Tracker}
website\cite{enforcementtrackerGDPREnforcement}.

This research is following the protocol of \textit{Prisma flowchart} for
systematic reviews\cite{prismaFlowchart}.

\textit{Google Scholar} was used to search for articles containing the keywords
above. The first 360 results were considered. The articles were filtered by
relevance following their titles and 60 were considered. The abstracts of the 60
articles were read and X were considered
% TODO: Describe flowchart with sentences above here TODO: Put numbers and add
% flowchart here as Figure 1

\subsection*{Inclusion criteria}

\begin{itemize}
    \item Articles published in English
    \item Articles published between 2018 and 2023
    \item Articles that discuss GDPR
    \item Articles that discuss Meta Platforms, Inc.
    \item Data between 2021 and 2023
\end{itemize}

\subsection*{Exclusion criteria}

\begin{itemize}
    \item Articles published in languages other than English
    \item Articles published before 2018
    \item Articles that do not discuss GDPR
    \item Articles that do not discuss Meta Platforms, Inc.
    \item Articles that discuss about ethics
    \item Data before 2021
\end{itemize}

\subsection*{Ethical considerations}

This study is approved by Fontys University of Applied Sciences, Eindhoven, The
Netherlands. All data is collected from publicly available sources and no
personal data is collected. The author is not affiliated with Meta Platforms,
Inc. or any other company mentioned in this study.

\section*{Results}

\subsection*{Fines analysis}

%Explore the fines imposed on Meta Platforms by regulatory authorities due to
%GDPR violations. Examine the magnitude of the fines, both in absolute terms and
%in relation to the company's financial performance. Compare the fines to Meta
%Platforms' revenue, profits, or other financial indicators to assess their
%significance.

\subsection*{Legal expenses}

% Investigate the costs incurred by Meta Platforms in legal proceedings related
% to GDPR non-compliance. This includes expenses associated with hiring legal
% counsel, litigation, settlement agreements, and any ongoing legal obligations.
% Analyze the financial impact of these legal expenses on the company's
% profitability.

\subsection*{Investor Confidence and Stock Performance}

% Assess the impact of GDPR non-compliance on Meta Platforms' stock performance.
% Analyze stock price movements, market reactions to GDPR-related news, and
% changes in investor sentiment. Explore how GDPR issues have influenced
% investor confidence and the company's market capitalization.

\subsection*{Advertiser Reactions}

% Study the response of advertisers to GDPR non-compliance by Meta Platforms.
% Examine whether there has been a shift in advertiser behavior, such as reduced
% advertising spend, contract cancellations, or demands for stricter data
% protection measures. Analyze the financial implications of these reactions on
% Meta Platforms' advertising revenue.

\subsection*{User Base and User Engagement}

% Investigate whether GDPR non-compliance has had any impact on Meta Platforms'
% user base and user engagement metrics. Analyze changes in user growth rates,
% active user counts, or user activity patterns following GDPR-related
% incidents. Assess the potential financial consequences of any decline in user
% metrics.

\subsection*{Reputational Damage}

% Assess the financial impact of reputational damage resulting from GDPR
% non-compliance. Investigate whether negative media coverage, public backlash,
% or loss of user trust have affected Meta Platforms' brand value or customer
% perception. Analyze any associated costs, such as brand recovery campaigns or
% decreased user acquisition rates.

\bibliographystyle{vancouver}
\bibliography{refs}
\end{document}
